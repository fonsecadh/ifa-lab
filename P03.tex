\documentclass[11pt]{article}
\usepackage[T1]{fontenc}
\usepackage{lmodern}
\usepackage{parskip}
\usepackage[colorlinks=true,urlcolor=Blue,linkcolor=black,citecolor=black]{hyperref}
\usepackage{graphicx}
\usepackage{amsmath}
\usepackage[utf8]{inputenc}
\usepackage[spanish]{babel}
\usepackage{fancyhdr}
\usepackage{csquotes}
\usepackage{lastpage}
\usepackage{array}
\usepackage{listings}
\usepackage{color}
\definecolor{dkgreen}{rgb}{0,0.6,0}
\definecolor{gray}{rgb}{0.5,0.5,0.5}
\definecolor{mauve}{rgb}{0.58,0,0.82}
\usepackage[affil-it]{authblk}
\usepackage[activate={true,nocompatibility},final,tracking=true,kerning=true,spacing=true,factor=1100,stretch=10,shrink=10]{microtype}
\usepackage[hmargin=2cm,top=4cm,headheight=65pt,footskip=65pt]{geometry}
\usepackage{hyperref}
\usepackage{graphicx}
\usepackage{float}
\usepackage{tabularx}
\graphicspath{ {./screenshots/p03/} }

% Documento
\begin{document}

% Título
\title{IFA. Práctica de laboratorio 03}

\author{Hugo Fonseca Díaz \\ email \href{mailto:uo258318@uniovi.es}{uo258318@uniovi.es}}
\affil{Escuela de Ingeniería Informática. Universidad de Oviedo.}

\maketitle

% Ejercicio 1
\section{Ejercicio 1}
Se crea el caso en Autopsy con los datos solicitados.

\begin{figure}[H]
    \caption{Ejercicio 1: Creación del caso}
    \centering
    \includegraphics[scale=0.7]{e1-1.png}
\end{figure}

\begin{figure}[H]
    \caption{Ejercicio 1: Detalles del examinador}
    \centering
    \includegraphics[scale=0.7]{e1-2.png}
\end{figure}

Añadimos la imagen a analizar.

\begin{figure}[H]
    \caption{Ejercicio 1: Selección de la imagen}
    \centering
    \includegraphics[scale=0.7]{e1-3.png}
\end{figure}

Se seleccionan los módulos de identificación de tipos de fichero, parseador de Exif y \textit{PhotoRec Carver}.

\begin{figure}[H]
    \caption{Ejercicio 1: Selección de módulos}
    \centering
    \includegraphics[scale=0.7]{e1-4.png}
\end{figure}

Se ejecuta el análisis y se obtienen los resultados con los que se rellenará la tabla.

\begin{figure}[H]
    \caption{Ejercicio 1: Resultados del análisis}
    \centering
    \includegraphics[scale=0.7]{e1-5.png}
\end{figure}

\begin{figure}[H]
    \caption{Ejercicio 1: Imágenes obtenidas}
    \centering
    \includegraphics[scale=0.7]{e1-6.png}
\end{figure}

Para abrir el archivo con extensión \textit{pcx} se ha utilizado un visor de imágenes online, al no disponer de uno adecuado en el equipo.

\begin{table}[H]
    \centering
    \begin{tabular}{|c|c|c|}
        \hline
        Nombre del fichero en Autopsy & Tamaño del fichero (en Bytes) & Breve descripción imagen visible \\
        \hline\hline
        f0010000.png & 12966639 & Flor morada \\
        \hline
        f0045326.pcx & 5926912 & Iglesia y fuente \\
        \hline
        f0056902.jpg & 60716 & Cartel 'Welcome to Moscow' \\
        \hline
        f0067021.bmp & 5111906 & Piedras en forma de Pi \\
        \hline
        f0087006.tif & 20089344 & Cañas de bambú \\
        \hline
        f0126243.gif & 132948 & Girasol \\
        \hline
    \end{tabular}
\end{table}

% Ejercicio 2
\section{Ejercicio 2}
Se crea el caso en Autopsy con los datos solicitados.

\begin{figure}[H]
    \caption{Ejercicio 2: Creación del caso}
    \centering
    \includegraphics[scale=0.7]{e2-1.png}
\end{figure}

\begin{figure}[H]
    \caption{Ejercicio 2: Detalles del examinador}
    \centering
    \includegraphics[scale=0.7]{e2-2.png}
\end{figure}

Añadimos la imagen a analizar.

\begin{figure}[H]
    \caption{Ejercicio 2: Selección de la imagen}
    \centering
    \includegraphics[scale=0.7]{e2-3.png}
\end{figure}

Se seleccionan los módulos de identificación de tipos de fichero, parseador de Exif y \textit{PhotoRec Carver}.

\begin{figure}[H]
    \caption{Ejercicio 2: Selección de módulos}
    \centering
    \includegraphics[scale=0.7]{e2-4.png}
\end{figure}

Se ejecuta el análisis y se obtienen los resultados con los que se rellenará la tabla.

\begin{figure}[H]
    \caption{Ejercicio 2: Resultados del análisis}
    \centering
    \includegraphics[scale=0.7]{e2-5.png}
\end{figure}

\begin{figure}[H]
    \caption{Ejercicio 2: Imágenes obtenidas}
    \centering
    \includegraphics[scale=0.7]{e2-6.png}
\end{figure}

Para abrir el archivo con extensión \textit{pcx} se ha utilizado un visor de imágenes online, al no disponer de uno adecuado en el equipo.

\begin{table}[H]
    \centering
    \begin{tabular}{|c|c|c|}
        \hline
        Nombre del fichero en Autopsy & Tamaño del fichero (en Bytes) & Breve descripción imagen visible \\
        \hline\hline
        f0010000.jpg & 60716 & Cartel 'Welcome to Moscow' \\
        \hline
        f0020119.tif & 20089344 & Cañas de bambú \\
        \hline
        f0059356.bmp & 5111906 & Piedras en forma de Pi \\
        \hline
        f0079341.pcx & 5926912 & Iglesia y fuente \\
        \hline
        f0090917.gif & 132948 & Girasol \\
        \hline
        f0101177.png & 12966639 & Flor morada \\
        \hline
    \end{tabular}
\end{table}

% Ejercicio 3
\section{Ejercicio 3}
Se crea el caso en Autopsy con los datos solicitados.

\begin{figure}[H]
    \caption{Ejercicio 3: Creación del caso}
    \centering
    \includegraphics[scale=0.7]{e3-1.png}
\end{figure}

\begin{figure}[H]
    \caption{Ejercicio 3: Detalles del examinador}
    \centering
    \includegraphics[scale=0.7]{e3-2.png}
\end{figure}

Añadimos la imagen a analizar.

\begin{figure}[H]
    \caption{Ejercicio 3: Selección de la imagen}
    \centering
    \includegraphics[scale=0.7]{e3-3.png}
\end{figure}

Se seleccionan los módulos de identificación de tipos de fichero, parseador de Exif y \textit{PhotoRec Carver}.

\begin{figure}[H]
    \caption{Ejercicio 3: Selección de módulos}
    \centering
    \includegraphics[scale=0.7]{e3-4.png}
\end{figure}

Se ejecuta el análisis y se obtienen los resultados con los que se rellenará la tabla.

\begin{figure}[H]
    \caption{Ejercicio 3: Resultados del análisis}
    \centering
    \includegraphics[scale=0.7]{e3-5.png}
\end{figure}

Para obtener las fechas se abren los documentos con las aplicaciones externas correspondientes y se busca en sus propiedades.

\begin{figure}[H]
    \caption{Ejercicio 3: Propiedades del pdf}
    \centering
    \includegraphics[scale=0.7]{e3-6.png}
\end{figure}

\begin{table}[H]
    \centering
    \begin{tabular}{|p{3cm}|p{2cm}|p{8cm}|p{2cm}|}
        \hline
        Nombre del fichero en Autopsy & Tamaño del fichero (en Bytes) & Tipo MIME documento & Fecha Creación del documento \\
        \hline\hline
        f0010000.pdf & 3176275 & application/pdf & 2008/06/06 \\
        \hline
        f0026204.pdf & 2525414 & application/pdf & 2008/06/04 \\
        \hline
        f0041137.xlsx & 23513 & application/vnd.openxmlformats-officedocument.spreadsheetml.sheet & 2012/06/13 \\
        \hline
        f0051183.xlsx & 13824 & application/vnd.openxmlformats-officedocument.spreadsheetml.sheet & 2012/07/05 \\
        \hline
        f0061210.docx & 4424 & application/vnd.openxmlformats-officedocument.wordprocessingml.document & Sin especificar \\
        \hline
        f0071219.docx & 4004 & application/vnd.openxmlformats-officedocument.wordprocessingml.document & Sin especificar \\
        \hline
        f0081227.pptx & 902645 & application/vnd.openxmlformats-officedocument.presentationml.presentation & 2010/09/28 \\
        \hline
    \end{tabular}
\end{table}

% Ejercicio 4
\section{Ejercicio 4}
Se crea el caso en Autopsy con los datos solicitados.

\begin{figure}[H]
    \caption{Ejercicio 4: Creación del caso}
    \centering
    \includegraphics[scale=0.7]{e4-1.png}
\end{figure}

\begin{figure}[H]
    \caption{Ejercicio 4: Detalles del examinador}
    \centering
    \includegraphics[scale=0.7]{e4-2.png}
\end{figure}

Añadimos la imagen a analizar.

\begin{figure}[H]
    \caption{Ejercicio 4: Selección de la imagen}
    \centering
    \includegraphics[scale=0.7]{e4-3.png}
\end{figure}

Se seleccionan los módulos de identificación de tipos de fichero, parseador de Exif y \textit{PhotoRec Carver}.

\begin{figure}[H]
    \caption{Ejercicio 4: Selección de módulos}
    \centering
    \includegraphics[scale=0.7]{e4-4.png}
\end{figure}

Se ejecuta el análisis y se obtienen los resultados con los que se rellenará la tabla.

\begin{figure}[H]
    \caption{Ejercicio 4: Resultados del análisis}
    \centering
    \includegraphics[scale=0.7]{e4-5.png}
\end{figure}

Para obtener las fechas se abren los documentos con las aplicaciones externas correspondientes y se busca en sus propiedades.

\begin{figure}[H]
    \caption{Ejercicio 4: Propiedades del pdf}
    \centering
    \includegraphics[scale=0.7]{e4-6.png}
\end{figure}

\begin{table}[H]
    \centering
    \begin{tabular}{|p{3cm}|p{2cm}|p{8cm}|p{2cm}|}
        \hline
        Nombre del fichero en Autopsy & Tamaño del fichero (en Bytes) & Tipo MIME documento & Fecha Creación del documento \\
        \hline\hline
        f0010000.pdf & 3176275 & application/pdf & 2008/06/06 \\
        \hline
        f0026204.pdf & 2525414 & application/pdf & 2008/06/04 \\
        \hline
        f0041137.xlsx & 23513 & application/vnd.openxmlformats-officedocument.spreadsheetml.sheet & 2012/06/13 \\
        \hline
        f0051183.xlsx & 13824 & application/vnd.openxmlformats-officedocument.spreadsheetml.sheet & 2012/07/05 \\
        \hline
        f0061210.docx & 4424 & application/vnd.openxmlformats-officedocument.wordprocessingml.document & Sin especificar \\
        \hline
        f0071219.docx & 4004 & application/vnd.openxmlformats-officedocument.wordprocessingml.document & Sin especificar \\
        \hline
        f0081227.pptx & 902645 & application/vnd.openxmlformats-officedocument.presentationml.presentation & 2010/09/28 \\
        \hline
    \end{tabular}
\end{table}

% Ejercicio 5
\section{Ejercicio 5}
Se crea el caso en Autopsy con los datos solicitados.

\begin{figure}[H]
    \caption{Ejercicio 5: Creación del caso}
    \centering
    \includegraphics[scale=0.7]{e5-1.png}
\end{figure}

\begin{figure}[H]
    \caption{Ejercicio 5: Detalles del examinador}
    \centering
    \includegraphics[scale=0.7]{e5-2.png}
\end{figure}

Añadimos la imagen a analizar.

\begin{figure}[H]
    \caption{Ejercicio 5: Selección de la imagen}
    \centering
    \includegraphics[scale=0.7]{e5-3.png}
\end{figure}

Se seleccionan los módulos de identificación de tipos de fichero, parseador de Exif y \textit{PhotoRec Carver}.

\begin{figure}[H]
    \caption{Ejercicio 5: Selección de módulos}
    \centering
    \includegraphics[scale=0.7]{e5-4.png}
\end{figure}

Se ejecuta el análisis y se obtienen los resultados con los que se responderá a las preguntas.

\begin{figure}[H]
    \caption{Ejercicio 5: Resultados del análisis}
    \centering
    \includegraphics[scale=0.7]{e5-5.png}
\end{figure}

a) Hay 71 falsos positivos.

b) Todos son de tipo texto plano.

Esto puede deberse a que Autopsy no haya sido capaz de recuperar los archivos con sus verdaderos tipos MIME y los fragmentos de esos archivos sean tratados como texto plano.

% Ejercicio 6
\section{Ejercicio 6}
Se crea el caso en Autopsy con los datos solicitados.

\begin{figure}[H]
    \caption{Ejercicio 6: Creación del caso}
    \centering
    \includegraphics[scale=0.7]{e6-1.png}
\end{figure}

\begin{figure}[H]
    \caption{Ejercicio 6: Detalles del examinador}
    \centering
    \includegraphics[scale=0.7]{e6-2.png}
\end{figure}

Añadimos la imagen a analizar.

\begin{figure}[H]
    \caption{Ejercicio 6: Selección de la imagen}
    \centering
    \includegraphics[scale=0.7]{e6-3.png}
\end{figure}

Se seleccionan los módulos de identificación de tipos de fichero, parseador de Exif, \textit{PhotoRec Carver} y el módulo de extracción de ficheros.

\begin{figure}[H]
    \caption{Ejercicio 6: Selección de módulos}
    \centering
    \includegraphics[scale=0.7]{e6-4.png}
\end{figure}

Se ejecuta el análisis y se obtienen los resultados con los que se rellenará la tabla.

\begin{figure}[H]
    \caption{Ejercicio 6: Resultados del análisis}
    \centering
    \includegraphics[scale=0.7]{e6-5.png}
\end{figure}

\begin{table}[H]
    \centering
    \begin{tabular}{|c|c|c|}
        \hline
        Nombre del fichero en Autopsy & Tamaño del fichero (en Bytes) & Tipo MIME \\
        \hline\hline
        f0010000.7z & 1130244 & application/x-7z-compressed \\
        \hline
        f0022208.bz2 & 5685248 & application/x-bzip2 \\
        \hline
        f0033312.gz & 5171200 & application/x-gzip \\
        \hline
        f0055780.wim & 941059 & application/octet-stream \\
        \hline
        f0067619.rar & 868785 & application/x-rar-compressed \\
        \hline
        f0079316.zip & 643764 & application/zip \\
        \hline
    \end{tabular}
\end{table}

Se extraen las imágenes de los ficheros comprimidos.

\begin{figure}[H]
    \caption{Ejercicio 6: Imágenes extraídas}
    \centering
    \includegraphics[scale=0.7]{e6-6.png}
\end{figure}

Se ejecuta la herramienta \textit{exiftool} para obtener los datos que se usan a la hora de rellenar la siguiente tabla.

\begin{figure}[H]
    \caption{Ejercicio 6: Resultado del comando \textit{exiftool} con una de las imágenes extraídas}
    \centering
    \includegraphics[scale=0.7]{e6-7.png}
\end{figure}

\begin{table}[H]
    \centering
    \begin{tabular}{|p{3cm}|p{2cm}|p{8cm}|p{3cm}|}
        \hline
        Nombre del fichero en Autopsy & Fecha y hora de la imagen & Dispositivo con el que se tomó la imagen & Breve descripción de la imagen \\
        \hline\hline
        6-f0033312 & 2003/02/18 10:46:51 & SR86 EPSON DIGITAL STILL CAMERA & Casa con nieve \\
        \hline
        5-f0022208 & 2004/05/31 15:03:51 & KODAK DX4530 ZOOM DIGITAL CAMERA & Flor blanca \\
        \hline
        100\_0094.jpg & 2004/06/19 04:52:06 & KODAK DX4530 ZOOM DIGITAL CAMERA & Iglesia / Catedral \\
        \hline
        100\_0172.jpg & 2004/07/02 19:42:41 & KODAK DX4530 ZOOM DIGITAL CAMERA & Flor rojiza \\
        \hline
        100\_0183.jpg & 2004/07/05 19:57:36 & KODAK DX4530 ZOOM DIGITAL CAMERA & Girasol \\
        \hline
        100\_0221.jpg & 2004/08/28 07:32:22 & KODAK DX4530 ZOOM DIGITAL CAMERA & Cañas bambú \\
        \hline
    \end{tabular}
\end{table}

% Ejercicio 7
\section{Ejercicio 7}
Se crea el caso en Autopsy con los datos solicitados.

\begin{figure}[H]
    \caption{Ejercicio 7: Creación del caso}
    \centering
    \includegraphics[scale=0.7]{e7-1.png}
\end{figure}

\begin{figure}[H]
    \caption{Ejercicio 7: Detalles del examinador}
    \centering
    \includegraphics[scale=0.7]{e7-2.png}
\end{figure}

Añadimos la imagen a analizar.

\begin{figure}[H]
    \caption{Ejercicio 7: Selección de la imagen}
    \centering
    \includegraphics[scale=0.7]{e7-3.png}
\end{figure}

Se seleccionan los módulos de identificación de tipos de fichero, parseador de Exif, \textit{PhotoRec Carver} y el módulo de extracción de ficheros.

\begin{figure}[H]
    \caption{Ejercicio 7: Selección de módulos}
    \centering
    \includegraphics[scale=0.7]{e7-4.png}
\end{figure}

Se ejecuta el análisis y se observa que Autopsy ha detectado una posible bomba zip entre uno de los ficheros comprimidos.

\begin{figure}[H]
    \caption{Ejercicio 7: Resultados del análisis}
    \centering
    \includegraphics[scale=0.7]{e7-5.png}
\end{figure}

Se instalan los paquetes \textit{clamav} y \textit{clamtk} y se analiza la carpeta donde se extrayeron los ficheros comprimidos.

\begin{figure}[H]
    \caption{Ejercicio 7: Instalación de \textit{clamav} y \textit{clamtk}}
    \centering
    \includegraphics[scale=0.7]{e7-6.png}
\end{figure}

\begin{figure}[H]
    \caption{Ejercicio 7: Resultados del análisis del antivirus}
    \centering
    \includegraphics[scale=0.7]{e7-7.png}
\end{figure}

a) El antivirus no detecta nada, pero Autopsy si que notificó que uno de los ficheros podía ser una bomba zip. Este es un ataque que comprime con una alta proporción una gran cantidad de datos en un archivo comprimido de pocos datos. Sirve para inutilizar los programas que descomprimem dicho fichero, normalmente se busca inutilizar un antivirus, para luego ejecutar otro tipo de malware.

b) Bomba zip.

% Ejercicio 8
\section{Ejercicio 8}
Se crea el caso en Autopsy con los datos solicitados.

\begin{figure}[H]
    \caption{Ejercicio 8: Creación del caso}
    \centering
    \includegraphics[scale=0.7]{e8-1.png}
\end{figure}

\begin{figure}[H]
    \caption{Ejercicio 8: Detalles del examinador}
    \centering
    \includegraphics[scale=0.7]{e8-2.png}
\end{figure}

Añadimos la imagen a analizar.

\begin{figure}[H]
    \caption{Ejercicio 8: Selección de la imagen}
    \centering
    \includegraphics[scale=0.7]{e8-3.png}
\end{figure}

Se seleccionan los módulos de identificación de tipos de fichero, parseador de Exif y \textit{PhotoRec Carver}.

\begin{figure}[H]
    \caption{Ejercicio 8: Selección de módulos}
    \centering
    \includegraphics[scale=0.7]{e8-4.png}
\end{figure}

Para rellenar la tabla se usarán los datos obtenidos al ejecutar el análisis de Autopsy y mediante el uso de la herramienta \textit{MediaInfo}.

\begin{figure}[H]
    \caption{Ejercicio 8: Resultados del análisis}
    \centering
    \includegraphics[scale=0.7]{e8-5.png}
\end{figure}

\begin{figure}[H]
    \caption{Ejercicio 8: Herramienta \textit{MediaInfo}}
    \centering
    \includegraphics[scale=0.7]{e8-6.png}
\end{figure}

\begin{table}[H]
    \centering
    \begin{tabular}{|p{2.5cm}|p{2cm}|c|p{1.5cm}|c|c|c|}
        \hline
        Nombre del fichero en Autopsy & Tamaño del fichero (en Bytes) & Tipo MIME & Autor & Género & Duración & Tasa de Muestreo \\
        \hline\hline
        f0010000.mp3 & 1053174 & audio/mpeg & Kevin McLeod & Electronica & 32s 783ms & 44.1kHz \\
        \hline
        f0022057.wav & 4612660 & audio/vnd.wave & - & - & 26s 148ms & 44.1kHz \\
        \hline
        f0041067.au & 9243672 & audio/basic & - & - & 3min 29s & 44.1kHz \\
        \hline
        f0069122.wma & 1074873 & audio/x-ms-wma & - & (80) & 1min 5s & 44.1kHz \\
        \hline
    \end{tabular}
\end{table}

% Ejercicio 9
\section{Ejercicio 9}
Se crea el caso en Autopsy con los datos solicitados.

\begin{figure}[H]
    \caption{Ejercicio 9: Creación del caso}
    \centering
    \includegraphics[scale=0.7]{e9-1.png}
\end{figure}

\begin{figure}[H]
    \caption{Ejercicio 9: Detalles del examinador}
    \centering
    \includegraphics[scale=0.7]{e9-2.png}
\end{figure}

Añadimos la imagen a analizar.

\begin{figure}[H]
    \caption{Ejercicio 9: Selección de la imagen}
    \centering
    \includegraphics[scale=0.7]{e9-3.png}
\end{figure}

Se seleccionan los módulos de identificación de tipos de fichero, parseador de Exif y \textit{PhotoRec Carver}.

\begin{figure}[H]
    \caption{Ejercicio 9: Selección de módulos}
    \centering
    \includegraphics[scale=0.7]{e9-4.png}
\end{figure}

Para rellenar la tabla se usarán los datos obtenidos al ejecutar el análisis de Autopsy y mediante el uso de la herramienta \textit{MediaInfo}.

\begin{figure}[H]
    \caption{Ejercicio 9: Resultados del análisis}
    \centering
    \includegraphics[scale=0.7]{e9-5.png}
\end{figure}

\begin{figure}[H]
    \caption{Ejercicio 9: Herramienta \textit{MediaInfo}}
    \centering
    \includegraphics[scale=0.7]{e9-6.png}
\end{figure}

\begin{table}[H]
    \centering
    \begin{tabular}{|p{2.5cm}|p{2cm}|c|p{1.5cm}|c|c|c|}
        \hline
        Nombre del fichero en Autopsy & Tamaño del fichero (en Bytes) & Tipo MIME & Autor & Género & Duración & Tasa de Muestreo \\
        \hline\hline
        f0010020.mp3 & 516598 & audio/mpeg & - & - & 16s 143ms & 44.1kHz \\
        \hline
        f0011029.mp3 & 526545 & audio/mpeg & Kevin McLeod & Electronica & 16s 326ms & 44.1kHz \\
        \hline
        f0022057.wav & 4612660 & audio/vnd.wave & - & - & 26s 148ms & 44.1kHz \\
        \hline
        f0047085.au & 9243672 & audio/basic & - & - & 3min 29s & 44.1kHz \\
        \hline
    \end{tabular}
\end{table}

Se puede observar que los dos ficheros mp3 son dos fragmentos del fichero mp3 del ejercicio 8.

% Ejercicio 10
\section{Ejercicio 10}
Se crea el caso en Autopsy con los datos solicitados.

\begin{figure}[H]
    \caption{Ejercicio 10: Creación del caso}
    \centering
    \includegraphics[scale=0.7]{e10-1.png}
\end{figure}

\begin{figure}[H]
    \caption{Ejercicio 10: Detalles del examinador}
    \centering
    \includegraphics[scale=0.7]{e10-2.png}
\end{figure}

Añadimos la imagen a analizar.

\begin{figure}[H]
    \caption{Ejercicio 10: Selección de la imagen}
    \centering
    \includegraphics[scale=0.7]{e10-3.png}
\end{figure}

Se seleccionan los módulos de identificación de tipos de fichero, parseador de Exif y \textit{PhotoRec Carver}.

\begin{figure}[H]
    \caption{Ejercicio 10: Selección de módulos}
    \centering
    \includegraphics[scale=0.7]{e10-4.png}
\end{figure}

Para rellenar la tabla se usarán los datos obtenidos al ejecutar el análisis de Autopsy y mediante el uso de las herramientas \textit{MediaInfo} y \textit{FileInfo}.

\begin{figure}[H]
    \caption{Ejercicio 10: Resultados del análisis}
    \centering
    \includegraphics[scale=0.7]{e10-5.png}
\end{figure}

\begin{figure}[H]
    \caption{Ejercicio 10: Herramienta \textit{FileInfo}}
    \centering
    \includegraphics[scale=0.7]{e10-6.png}
\end{figure}

\begin{table}[H]
    \centering
    \begin{tabular}{|p{2.5cm}|p{2cm}|c|c|c|p{1.5cm}|c|p{2cm}|}
        \hline
        Nombre del fichero en Autopsy & Tamaño del fichero (en Bytes) & Tipo MIME & FPS & Resolución & Tasa Muestreo & Duración & Fecha \\
        \hline\hline
        f001000.mp4 & 1136007 & video/mp4 & 24 & 512x340 & 24 bit depth & 16.38s & 2012/07/17 18:59:19 \\
        \hline
        f0022219.avi & 1470360 & video/x-msvideo & 30 & 512x288 & 44kHz & 10.13s & - \\
        \hline
        f0035091.mov & 497414 & video/quicktime & 29.97 & 512x288 & 24 bit depth & 12.15s & 2012/08/02 13:19:44 \\
        \hline
        f0046063.flv & 3713213 & video/x-flv & 30 & 640x480 & 44kHz & 26.07s & - \\
        \hline
        f0063316.mpg & 22579200 & video/mpeg & 29.97 & 512x288 & 44kHz & 19.79s & - \\
        \hline
        f00117416.wmv & 2461696 & video/x-ms-wmv & 29.97 & 512x288 & 44kHz & 25.53s & 2012/08/16 12:10:37 \\
        \hline
    \end{tabular}
\end{table}

% Ejercicio 11
\section{Ejercicio 11}
Para realizar la primera parte del ejercicio se utilizará el comando \verb|xxd| junto al comando \verb|grep|. Empezamos buscando la cadena \textit{JFIF} en la imagen del ejercicio. Con esta búsqueda se sacará el offset en hexadecimal, y se convertirá a decimal en \textit{bash}.

\begin{figure}[H]
    \caption{Ejercicio 11: \textit{xxd} con una pipe a \textit{grep}}
    \centering
    \includegraphics[scale=0.7]{e11-1.png}
\end{figure}

Una vez obtenido el offset, se buscará el final de la imagen. Para ello se buscará con \verb|grep| la cadena \textit{ffd9} pasándole al comando \verb|xxd| el offset obtenido previamente.

\begin{figure}[H]
    \caption{Ejercicio 11: Buscando el final de la imagen}
    \centering
    \includegraphics[scale=0.7]{e11-2.png}
\end{figure}

Una vez se ha encontrado el offset del final de la imagen, se le suma el desplazamiento, en este caso 10 bytes, y se calcula el tamaño restando el offset final del inicial.

\begin{figure}[H]
    \caption{Ejercicio 11: Calculando tamaño de imagen}
    \centering
    \includegraphics[scale=0.7]{e11-3.png}
\end{figure}

Ahora se utiliza el comando \verb|dd| con la información obtenida previamente y se comprueba que la imagen extraída es la del cartel que pone 'Welcome to Moscow'.

\begin{figure}[H]
    \caption{Ejercicio 11: Carving de la imagen}
    \centering
    \includegraphics[scale=0.7]{e11-4.png}
\end{figure}

% Ejercicio 12
\section{Ejercicio 12}
Se crea el caso en Autopsy con los datos solicitados.

\begin{figure}[H]
    \caption{Ejercicio 12: Creación del caso}
    \centering
    \includegraphics[scale=0.7]{e12-1.png}
\end{figure}

\begin{figure}[H]
    \caption{Ejercicio 12: Detalles del examinador}
    \centering
    \includegraphics[scale=0.7]{e12-2.png}
\end{figure}

Añadimos la imagen a analizar.

\begin{figure}[H]
    \caption{Ejercicio 12: Selección de la imagen}
    \centering
    \includegraphics[scale=0.7]{e12-3.png}
\end{figure}

Se seleccionan los módulos de identificación de tipos de fichero y \textit{PhotoRec Carver}.

\begin{figure}[H]
    \caption{Ejercicio 12: Selección de módulos}
    \centering
    \includegraphics[scale=0.7]{e12-4.png}
\end{figure}

Se obtienen los resultados del análisis con los que se responderán a las diferentes cuestiones del ejercicio.

\begin{figure}[H]
    \caption{Ejercicio 12: Resultados del análisis}
    \centering
    \includegraphics[scale=0.7]{e12-5.png}
\end{figure}

a)

\begin{table}[H]
    \centering
    \begin{tabular}{|c|c|c|c|}
        \hline
        Número partición & Sector comienzo & Sector finalización & Tipo Sistema de Ficheros \\
        \hline\hline
        1 & 0 & 127 & Unallocated \\
        \hline
        2 & 128 & 16511 & DOS FAT12 \\
        \hline
        3 & 16512 & 82047 & DOS FAT16 \\
        \hline
        4 & 82048 & 213119 & Win95 FAT32 \\
        \hline
        5 & 213120 & 2097152 & Unallocated \\
        \hline
    \end{tabular}
\end{table}

b) Para responder a esta cuestión se observan los resultados de la pestaña 'Views'.

\begin{figure}[H]
    \caption{Ejercicio 12: Ficheros de texto plano}
    \centering
    \includegraphics[scale=0.7]{e12-6.png}
\end{figure}

Se puede ver que hay 9 ficheros de texto plano, 3 de ellos borrados.

c) Para rellenar esta tabla se miran los metadatos que muestra Autopsy de cada archivo borrado.

\begin{figure}[H]
    \caption{Ejercicio 12: Metadatos de los ficheros borrados}
    \centering
    \includegraphics[scale=0.7]{e12-7.png}
\end{figure}

\begin{table}[H]
    \centering
    \begin{tabular}{|c|c|c|c|p{2cm}|p{2cm}|p{2cm}|}
        \hline
        Nombre & Tamaño & Partición & Sector relativo & Acceso (GMT) & Modificación (GMT) & Creación (GMT) \\
        \hline\hline
        \_BEID.txt & 712 & vol 2 & 170 & 1999/01/01 23:00:00 & 2000/02/29 13:11:00 & 2011/12/25 13:02:22 \\
        \hline
        Betelgeuse.txt & 712 & vol 3 & 546 & 1999/01/01 23:00:00 & 2000/02/29 13:12:00 & 2011/12/25 13:02:24 \\
        \hline
        Bellatrix.txt & 712 & vol 4 & 8195 & 1999/01/01 23:00:00 & 2000/02/29 13:13:00 & 2011/12/25 13:02:24 \\
        \hline
    \end{tabular}
\end{table}

d) Se muestran a continuación las líneas de tiempo de los tres ficheros borrados, en el filtro de la parte izquierda de la captura se observa el fichero actual.

\begin{figure}[H]
    \caption{Ejercicio 12: Línea temporal de \textit{Bellatrix.txt}}
    \centering
    \includegraphics[scale=0.7]{e12-8.png}
\end{figure}

\begin{figure}[H]
    \caption{Ejercicio 12: Línea temporal de \textit{\_BEID.txt}}
    \centering
    \includegraphics[scale=0.7]{e12-9.png}
\end{figure}

\begin{figure}[H]
    \caption{Ejercicio 12: Línea temporal de \textit{Betelgeuse.txt}}
    \centering
    \includegraphics[scale=0.7]{e12-10.png}
\end{figure}

% Ejercicio 13
\section{Ejercicio 13}
Se crea el caso en Autopsy con los datos solicitados.

\begin{figure}[H]
    \caption{Ejercicio 13: Creación del caso}
    \centering
    \includegraphics[scale=0.7]{e13-1.png}
\end{figure}

\begin{figure}[H]
    \caption{Ejercicio 13: Detalles del examinador}
    \centering
    \includegraphics[scale=0.7]{e13-2.png}
\end{figure}

Añadimos la imagen a analizar.

\begin{figure}[H]
    \caption{Ejercicio 13: Selección de la imagen}
    \centering
    \includegraphics[scale=0.7]{e13-3.png}
\end{figure}

Se seleccionan los módulos de identificación de tipos de fichero y \textit{PhotoRec Carver}.

\begin{figure}[H]
    \caption{Ejercicio 13: Selección de módulos}
    \centering
    \includegraphics[scale=0.7]{e13-4.png}
\end{figure}

Se obtienen los resultados del análisis con los que se responderán a las diferentes cuestiones del ejercicio.

\begin{figure}[H]
    \caption{Ejercicio 13: Resultados del análisis}
    \centering
    \includegraphics[scale=0.7]{e13-5.png}
\end{figure}

a)

TBD CAMBIAR TABLA!
\begin{table}[H]
    \centering
    \begin{tabular}{|c|c|c|c|}
        \hline
        Número partición & Sector comienzo & Sector finalización & Tipo Sistema de Ficheros \\
        \hline\hline
        1 & 0 & 127 & Unallocated \\
        \hline
        2 & 128 & 16511 & DOS FAT12 \\
        \hline
        3 & 16512 & 82047 & DOS FAT16 \\
        \hline
        4 & 82048 & 213119 & Win95 FAT32 \\
        \hline
        5 & 213120 & 2097152 & Unallocated \\
        \hline
    \end{tabular}
\end{table}

b) Para responder a esta cuestión se observan los resultados de la pestaña 'Views'.

\begin{figure}[H]
    \caption{Ejercicio 13: Ficheros de texto plano}
    \centering
    \includegraphics[scale=0.7]{e13-6.png}
\end{figure}

Se puede ver que hay 9 ficheros de texto plano. Hay 4 ficheros adicionales borrados, uno llamado OrphanFiles, el cual es autogenerado por Autopsy, y tres ficheros con extensión txt pero cuyos tipos MIME no son texto plano.

c) Para rellenar esta tabla se miran los metadatos que muestra Autopsy de cada archivo borrado.

\begin{figure}[H]
    \caption{Ejercicio 13: Metadatos de los ficheros borrados}
    \centering
    \includegraphics[scale=0.7]{e13-7.png}
\end{figure}

Como se puede observar hay menos metadatos sobre los ficheros borrados que en el ejercicio anterior, por lo que habrá secciones de la tabla sin rellenar.

TBD CAMBIAR TABLA!
\begin{table}[H]
    \centering
    \begin{tabular}{|c|c|c|c|p{2cm}|p{2cm}|p{2cm}|}
        \hline
        Nombre & Tamaño & Partición & Sector relativo & Acceso (GMT) & Modificación (GMT) & Creación (GMT) \\
        \hline\hline
        \_BEID.txt & 712 & vol 2 & 170 & 1999/01/01 23:00:00 & 2000/02/29 13:11:00 & 2011/12/25 13:02:22 \\
        \hline
        Betelgeuse.txt & 712 & vol 3 & 546 & 1999/01/01 23:00:00 & 2000/02/29 13:12:00 & 2011/12/25 13:02:24 \\
        \hline
        Bellatrix.txt & 712 & vol 4 & 8195 & 1999/01/01 23:00:00 & 2000/02/29 13:13:00 & 2011/12/25 13:02:24 \\
        \hline
    \end{tabular}
\end{table}

d) Se muestran a continuación las líneas de tiempo de los tres ficheros borrados, en el filtro de la parte izquierda de la captura se observa el fichero actual.

\begin{figure}[H]
    \caption{Ejercicio 13: Línea temporal de \textit{Bellatrix.txt}}
    \centering
    \includegraphics[scale=0.7]{e13-8.png}
\end{figure}

Se observa que no hay datos para \textit{Bellatrix.txt}

\begin{figure}[H]
    \caption{Ejercicio 13: Línea temporal de \textit{Bunda.txt}}
    \centering
    \includegraphics[scale=0.7]{e13-9.png}
\end{figure}

\begin{figure}[H]
    \caption{Ejercicio 13: Línea temporal de \textit{Botein.txt}}
    \centering
    \includegraphics[scale=0.7]{e13-10.png}
\end{figure}

Para \textit{Bunda.txt} y \textit{Botein.txt} sí que se recuperan datos.

% Ejercicio 14
\section{Ejercicio 14}
Se responde a continuación a las diferentes cuestiones planteadas por el ejercicio.

a) Se utiliza el comando \verb|mmls|, que lista las particiones con sus sectores de inicio y fin, entre otros datos.

\begin{figure}[H]
    \caption{Ejercicio 14: Salida del comando \textit{mmls}}
    \centering
    \includegraphics[scale=0.7]{e14-1.png}
\end{figure}

b) Sí, la información es consistente entre ambas herramientas.

c) Se usa el comando \verb|fsstat|, con la flag \textit{t} para mostrar solo el tipo de partición y la flag \textit{o} para pasarle al comando el sector donde comienza la partición.

\begin{figure}[H]
    \caption{Ejercicio 14: Salida del comando \textit{fsstat} para las diferentes particiones}
    \centering
    \includegraphics[scale=0.7]{e14-2.png}
\end{figure}

d) Se utiliza el comando \verb|fls| que recibe como argumentos, entre otros, el comienzo del sector de la partición que se quiere analizar.

\begin{figure}[H]
    \caption{Ejercicio 14: Salida del comando \textit{fls} con las flags \textit{ro}}
    \centering
    \includegraphics[scale=0.7]{e14-3.png}
\end{figure}

e) Se usa ahora el comando \verb|fls| con las flags \textit{dFro}, \textit{d} muestra solo elementos borrados, \textit{F} muestra solo ficheros, \textit{r} es para que la búsqueda sea recursiva y \textit{o} para introducir el comienzo del sector de la partición.

\begin{figure}[H]
    \caption{Ejercicio 14: Salida del comando \textit{fls} con las flags \textit{dFro}}
    \centering
    \includegraphics[scale=0.7]{e14-4.png}
\end{figure}

f) Se utiliza el comando \verb|ffind| con las flags \textit{oa}, \textit{o} para introducir el comienzo del sector de la partición y \textit{a} para buscar todos los ficheros asociados. Se le pasa al comando el inodo del elemento que se está buscando, en este caso el 13.

\begin{figure}[H]
    \caption{Ejercicio 14: Salida del comando \textit{ffind}}
    \centering
    \includegraphics[scale=0.7]{e14-5.png}
\end{figure}

g) Se usa el comando \verb|istat| pasandole como argumento el comienzo del sector de la partición y el inodo a buscar.

\begin{figure}[H]
    \caption{Ejercicio 14: Salida del comando \textit{istat} para el inodo 13}
    \centering
    \includegraphics[scale=0.7]{e14-6.png}
\end{figure}

h) Se usa el comando \verb|istat| con la flag \textit{f} y el argumento \textit{list}.

\begin{figure}[H]
    \caption{Ejercicio 14: Salida del comando \textit{istat -f list}}
    \centering
    \includegraphics[scale=0.7]{e14-7.png}
\end{figure}



% Bibliografía
\begin{thebibliography}{8}
\end{thebibliography}

\end{document}


