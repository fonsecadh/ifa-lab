\documentclass[11pt]{article}
\usepackage[T1]{fontenc}
\usepackage{lmodern}
\usepackage{parskip}
\usepackage[colorlinks=true,urlcolor=Blue,linkcolor=black,citecolor=black]{hyperref}
\usepackage{graphicx}
\usepackage{amsmath}
\usepackage[utf8]{inputenc}
\usepackage[spanish]{babel}
\usepackage{fancyhdr}
\usepackage{csquotes}
\usepackage{lastpage}
\usepackage{array}
\usepackage{listings}
\usepackage{color}
\definecolor{dkgreen}{rgb}{0,0.6,0}
\definecolor{gray}{rgb}{0.5,0.5,0.5}
\definecolor{mauve}{rgb}{0.58,0,0.82}
\usepackage[affil-it]{authblk}
\usepackage[activate={true,nocompatibility},final,tracking=true,kerning=true,spacing=true,factor=1100,stretch=10,shrink=10]{microtype}
\usepackage[hmargin=2cm,top=4cm,headheight=65pt,footskip=65pt]{geometry}
\usepackage{hyperref}
\usepackage{graphicx}
\usepackage{float}
\graphicspath{ {./screenshots/p02/} }

% Documento
\begin{document}

% Título
\title{IFA. Práctica de laboratorio 02}

\author{Hugo Fonseca Díaz \\ email \href{mailto:uo258318@uniovi.es}{uo258318@uniovi.es}}
\affil{Escuela de Ingeniería Informática. Universidad de Oviedo.}

\maketitle

% Ejercicio 1
\section{Ejercicio 1}
Se guarda la fecha y hora del sistema en el archivo \verb|ej01.txt| con el comando \verb|date > ej01.txt|. Se muestra ese archivo con el comando \verb|cat|.

\begin{figure}[H]
  \caption{Ejercicio 1: Resultado del comando \textit{cat ej01.txt}.}
  \centering
  \includegraphics{e1-1.png}
\end{figure}

Se accede al sitio web \url{https://time.is/es/Spain} y se comprueba que la hora es la misma.

\begin{figure}[H]
    \caption{Ejercicio 1: Hora en el sitio web \textit{time.is}.}
  \centering
  \includegraphics{e1-2.png}
\end{figure}

% Ejercicio 2
\section{Ejercicio 2}
Se utiliza el comando \verb|uname| con las opciones \verb|v| (lista la versión del kernel) y \verb|o| (lista el nombre del sistema operativo).

\begin{figure}[H]
    \caption{Ejercicio 2: \textit{uname -vo}.}
  \centering
  \includegraphics{e2.png}
\end{figure}

% Ejercicio 3
\section{Ejercicio 3}
Se utiliza el comando \verb|lshw|, primero con la flag \verb|short| para encontrar el nombre de la clase de los dispositivos de red.

\begin{figure}[H]
    \caption{Ejercicio 3: \textit{lshw -short}.}
  \centering
  \includegraphics{e3-1.png}
\end{figure}

Una vez se sabe que el nombre de la clase de los dispositivos de red es \verb|network|, se utiliza el comando \verb|lshw| con la flag \verb|-class network|.

\begin{figure}[H]
    \caption{Ejercicio 3: \textit{lshw -class network}.}
  \centering
  \includegraphics{e3-2.png}
\end{figure}

También puede utilizarse el comando \verb|ip -h a| para mostrar más información sobre el dispositivo de red \verb|enp0s3|.

\begin{figure}[H]
    \caption{Ejercicio 3: \textit{ip -h a}.}
  \centering
  \includegraphics{e3-3.png}
\end{figure}

% Ejercicio 4
\section{Ejercicio 4}
Se utiliza el comando \verb|netstat| del paquete \verb|net-tools|. Su flag \verb|a| permite ver todos los sockets, por lo que \verb|sudo netstat -a > ej04.txt| guarda la información de los sockets activos y no activos en un fichero de texto. También son interesantes sus flags \verb|n| (se muestran las direcciones numéricamente), \verb|p| (se muestran los procesos pertenecientes a los sockets), \verb|t| (tcp) y \verb|u| (udp).

\begin{figure}[H]
    \caption{Ejercicio 4: \textit{cat ej04.txt | less}.}
  \centering
  \includegraphics{e4-1.png}
\end{figure}

\begin{figure}[H]
    \caption{Ejercicio 4: \textit{sudo netstat -ptun}.}
  \centering
  \includegraphics{e4-2.png}
\end{figure}

También se puede ver información de los servicios de red en \verb|/etc/services|.

\begin{figure}[H]
    \caption{Ejercicio 4: \textit{less /etc/services}.}
  \centering
  \includegraphics{e4-3.png}
\end{figure}

% Ejercicio 5
\section{Ejercicio 5}
Para resolver este ejercicio se usan tres comandos: \verb|who| muestra los usuarios conectados y la terminal en la que están, \verb|tty| muestra la terminal conectada actualmente al standard input y \verb|uptime| muestra el tiempo que ha pasado desde el arranque del sistema.

\begin{figure}[H]
    \caption{Ejercicio 5: \textit{who, tty} y \textit{uptime}.}
  \centering
  \includegraphics{e5.png}
\end{figure}

% Ejercicio 6
\section{Ejercicio 6}
Existen al menos dos opciones de mostrar la información sobre la tabla de enrutamiento: mediante el comando \verb|netstat| con su flag \verb|r| (que muestra la tabla de enrutamiento) o usando el comando \verb|route| con su flag \verb|n| (que muestra las direcciones de red de forma numérica).

\begin{figure}[H]
    \caption{Ejercicio 6: \textit{netstat -r} y \textit{route -n}.}
  \centering
  \includegraphics{e6.png}
\end{figure}

% Ejercicio 7
\section{Ejercicio 7}
Se usa el comando \verb|ps|. Dicho comando puede utilizarse siguiendo tres sintaxis: la de UNIX, la de BSD o la de GNU. Para mostrar todos los procesos del sistema con sintaxis de UNIX podría usarse \verb|ps -eF|. Con sintaxis de BSD se puede usar \verb|ps axu|. Para que se muestre el nombre del proceso sin cortarse se puede pasar el resultado del comando \verb|ps| al comando \verb|less| con una pipe de UNIX.

\begin{figure}[H]
    \caption{Ejercicio 7: \textit{ps axu | less}.}
  \centering
  \includegraphics{e7.png}
\end{figure}

% Ejercicio 8
\section{Ejercicio 8}
Se usarán los comandos \verb|last| y \verb|lastb|. El primero se utiliza para sacar la información de los accesos de todos los usuarios al sistema, incluyendo también un ejemplo de uso para un usuario concreto. El segundo es un comando similar pero buscando en \textit{/var/log/btmp}, lo que muestra intentos fallidos de acceso al sistema.

\begin{figure}[H]
    \caption{Ejercicio 8: \textit{last} y \textit{lastb}.}
  \centering
  \includegraphics{e8.png}
\end{figure}

% Ejercicio 9
\section{Ejercicio 9}
Se utiliza el comando \verb|lsof|, cuya salida está pensada para ser la entrada de otro programa que la parsee. Se hace una pipe de Unix con el comando \verb|less| para poder visualizar la salida del comando.

\begin{figure}[H]
    \caption{Ejercicio 9: \textit{lsof | less}.}
  \centering
  \includegraphics{e9.png}
\end{figure}

% Ejercicio 10
\section{Ejercicio 10}
Se puede usar el comando \verb|lsblk| con la opción \verb|f|. El comando muestra información de los dispositivos del sistema y la opción \textit{f} muestra los sistemas de ficheros de los mismos.

\begin{figure}[H]
    \caption{Ejercicio 10: \textit{lsblk -f}.}
  \centering
  \includegraphics{e10.png}
\end{figure}

% Ejercicio 11
\section{Ejercicio 11}
Para mostrar las particiones del disco \textit{sda} junto a sus sectores de inicio y fin, se utiliza el comando \verb|fdisk| con la opción \verb|l|, que lista dichas particiones, y pasándole como parámetro el disco que queremos inspeccionar (en este caso \textit{/dev/sda}). No es necesario especificarle que las unidades del tamaño sean sectores puesto que es el comportamiento por defecto.

\begin{figure}[H]
    \caption{Ejercicio 11: \textit{sudo fdisk -l /dev/sda}.}
  \centering
  \includegraphics{e11.png}
\end{figure}

% Bibliografía
\begin{thebibliography}{8}
\end{thebibliography}

\end{document}


